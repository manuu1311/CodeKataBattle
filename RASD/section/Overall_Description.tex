\definecolor{myred}{rgb}{0.99, 0.0, 0.0}
\section{Overall Description}

\subsection{Product perspective}
\subsubsection{Scenarios}
\begin{enumerate}[label=\textbf{\Alph*}.]
    \item \textbf{Registration} \\
        Emanuele is a Computer Science and Engineering student. He wants to showcase and develop his programming skills. He heard about CKB and decides to give it a try. He goes to the CKB website and creates an account providing his personal info, his academic status (student/educator),  and setting up a password. After that, he can join tournaments and finally start showing off his abilities while becoming a code master in the process.\\
    \item \textbf{Tournament creation} \\
        Matteo is a Google hiring manager. He is looking for new talented and enthusiastic software engineers. He creates a new tournament specifying the deadline for the subscription and grants his team permission to create new battles. Users who enabled notifications will receive an email notifying them about the creation of this new tournament. Matteo finds a lot of passionate coders and engages with them in a wide variety of battles. He can then pick those who performed brilliantly and can offer them a job at Google.\\
    \item \textbf{Battle creation} \\
        Giuseppe is a recruiter from Google with an 'educator' account in CKB. He has been granted permission to create battles  by his superior, Matteo. He publishes the code kata (that includes everything to make the battle work, from the description to test cases and automation scripts) and chooses edits the battle configuration settings such as minimum and maximum number of students per team, submission deadline and more. Students subscribed to this tournament can form teams and start working on the project. After the submission deadline, teams will receive an automatic evaluation. Giuseppe is allowed to review the evaluations if need be, and finally publish the final grades for each team.  \\ 
    \item \textbf{Badge creation} \\
        Matteo wants to add achievements to his tournament to hype up the students. He chooses a few conditions that he finds interesting selects the titles and creates the badges. At the end of the tournament, all the system checks the badges conditions to each user. Ultimately, the badges will be assigned to the students and they will be able to show them in their personal profile.
\end{enumerate}

\subsubsection{Class diagram}
The presented UML class diagram illustrates a conceptual, high-level model of the software. Due to its nature, it might model entities that won't necessarily be part of the final system under development. At this stage, it lacks references to methods and other low-level details, as these will be elaborated during the subsequent design phase.
\begin{figure}[H]
      \centering
      \includegraphics[scale=0.4]{src/class_uml.png}
\end{figure} \vspace{1cm}

The main entities are:
\begin{itemize}
    \item \textbf{User}\\
        A user can be either a student or an educator. Users can perform different actions according to this: educators create tournaments and battles, while students join them.
    \item \textbf{Tournament}\\
        Tournaments are created by educators and students can subscribe to them. They consist of different battles. The creator of the tournament can grant other educators privileges to define battles. Each student has a score consisting of the cumulative sum of scores obtained in all battless. Badges can also be assigned to a specific tournament.
    \item \textbf{Battles}\\
        Battles are created by educators with privileges in the scope of a tournament. Numerous settings are available for customization, such as the minimum and maximum settings for a team, deadlines and more. Students subscribed to a tournament need to form a team for each battle. At the end of the battle, each team is assigned a score between 0 and 100.
    \item \textbf{Badge}\\
        Badges are gamification objects that students can collect and show off in their profile. Educators can create badges and associate them to a specific tournament. Each badge has a set of rules that students need to satisfy in order to achieve It.
\end{itemize}

\subsubsection{State diagrams}
The purpose of the following UML state diagrams is to provide additional information about the behaviour of the system. These are the scenarios that we found more interesting.

\begin{figure}[H]
      \centering
      \includegraphics[scale=0.4]{src/state_diagrams/tournament_uml.png}
      \caption{Tournament timeline}
\end{figure} \vspace{1cm}
    This diagram describes what happens inside the system throughout a tournament, from the creation to the conclusion. At first, the system is waiting for new events. When an educator creates a new tournament, the system responds by sending notifications to all students. After that, the system enters a waiting state, allowing students to subscribe to the tournament until the specified deadline. When the tournament is finally open, authorized educators can create new battles. This process is seen more in-depth in the next diagram. After each battle, the system automatically updates the score for each student. New battles can be created until the creator closes the tournament. Students who took part in this tournament are sent a notification about their score, their ranking and eventual badges they may have achieved. At any moment, the creator can grant other educators permission to create battles in the scope of his tournament. \\

\begin{figure}[H]
      \centering
      \includegraphics[scale=0.4]{src/state_diagrams/battle_uml.png}
      \caption{Battle timeline}
\end{figure} \vspace{1cm}
    This diagrams describes the steps of a battle. When a new battle is created in the scope of a tournament, the system sends a notification to all students subscribed to that tournament. Students have some time to form teams and after the deadline, a github repository is automatically created. Teams can then start working on the project. The system is notified through API calls every time a team pushes new code. Following each push, the system downloads and evaluates the sources, then publishes the scores up to that point. The specifics of these calls are beyond the scope of this diagram. After the deadline, there is a consolidation phase, where authorized educators can manually adjust the scores if need be. Finally, the battle officially ends and the students are notified.

\subsection{Product Functions}
\subsubsection{User Registration and Authentication}
\begin{itemize}
    \item \textbf{User Registration}: Users can create accounts on the CKB platform by providing necessary information such as academic position (educator or student), email address and password. The registration process includes validation and email verification.
    \item \textbf{Authentication}: Once registered, users can log in with their credentials to access the platform. Authentication ensures that only authorized users can use the platform and the available functions are restricted depending on the user.
    \item \textbf{User Account Management}: Registered users have the ability to manage their account settings, update their profile information, and change passwords as needed.
\end{itemize}

\subsubsection{Creating and Managing Tournaments}
\begin{itemize}
    \item \textbf{Tournament Creation}: Educators, who act as administrators, can create new tournaments. They provide details about the tournament, including its name, description, duration, and any specific rules.
    \item \textbf{Tournament Management}: Educators can modify and manage existing tournaments. They can update tournament details, add or remove battles, and set deadlines for registration and submission. They can add other Educators to administer the tournament, too.
\end{itemize}

\subsubsection{Joining and Forming Teams}
\begin{itemize}
    \item \textbf{Joining Battles}: Students can browse and subscribe to available tournaments and join specific battles within those tournaments. They can view information about battles, such as the code kata description and participation rules.
    \item \textbf{Team Formation}: Students can form teams by inviting other students to join them. The system checks that they adhere to the minimum and maximum team size rules set by the educator.
\end{itemize}

\subsubsection{Code Kata Battles}
\begin{itemize}
    \item \textbf{Accessing Projects}: Students participating in a battle can download the respective project. Each project consists of a problem description and an initial codebase with missing functionality.
\end{itemize}
\subsubsection{GitHub Integration}
\begin{itemize}
    \item \textbf{GitHub Repository Creation}: For each code kata, the platform creates a corresponding GitHub repository for each team. This repository contains the initial codebase and testing scripts.
    \item \textbf{Updating and Testing Projects}: Students set up GitHub Actions to automate the submission process. With each code commit to their repository, GitHub Actions trigger updates and testing on the platform.
\end{itemize}
\subsubsection{Scoring and Ranking}
\begin{itemize}
    \item \textbf{Score Calculation}: The platform calculates scores based on the number of passed test cases, timeliness of submissions, and code quality. The higher the number of passing test cases, the higher the score.
    \item \textbf{Ranking System}: Students can view the real-time ranking of their teams and others in the same battle. Rankings are based on scores achieved during the competition.
\end{itemize}

\subsubsection{Consolidation and Manual Evaluation}
\begin{itemize}
    \item \textbf{Consolidation Stage}: After the submission deadline, educators review the code and manually evaluate aspects that cannot be determined automatically, such as code quality and innovation.
    \item \textbf{Final Ranking}: Once the consolidation stage is complete, the final ranking is determined, and students are notified of their positions and scores.
\end{itemize}
\subsubsection{Gamification and Badges}
\begin{itemize}
    \item \textbf{Badge Definition}: Educators can create gamification badges with specific titles and rules. 
    \item \textbf{Badge Assignment}: The badges are awarded to students based on the rules defined by the educators.
\end{itemize}
\subsubsection{Notification and Communication}
\begin{itemize}
    \item \textbf{Notification System}: The platform sends notifications to students about various events, such as new tournament announcements, battle updates, and final tournament rank confirmations.
\end{itemize}

\subsubsection{Profile and Tournament Statistics}
\begin{itemize}
    \item \textbf{User Profiles}: Each user has a profile that displays their personal information, tournament participation, and badges earned.
    \item \textbf{Tournament Statistics}: Users can view statistics related to their performance in each tournament, including scores, rankings, and tournament history.
\end{itemize}

\subsection{User characteristics}
There are mainly two kinds of users that interact with the system: Educator and Student.
\subsubsection{Educator}
\begin{itemize}
    \item \textbf{Teaching Level}: Educators can be university professors, instructors, or trainers responsible for organizing and supervising code kata battles.
    \item \textbf{Educational Institution}: They may work in educational institutions and have experience in teaching software development.
    \item \textbf{Platform Experience}: Educators should be familiar with the CodeKataBattle platform to create and manage tournaments, set up battles, and assess student performance.
    \item \textbf{Technical Proficiency}: While not all educators need to be technically proficient, some may have a background in software development and programming.
\end{itemize}
\subsubsection{Student}
\begin{itemize}
    \item \textbf{Programming Proficiency}: Students may vary in their level of programming proficiency, ranging from beginners to advanced developers.
    \item \textbf{Educational Level}: Students from various educational backgrounds, such as undergraduate, graduate, or coding boot camp participants, may use the platform.
    \item Motivation: Students may use the platform for different reasons, including academic requirements, self-improvement, or competition.
    \item \textbf{Preferred Programming Languages}: Users may have preferences for specific programming languages and may select code katas accordingly.
    \item \textbf{Team Formation}: Students can either participate individually or form teams based on the requirements of specific battles.
    \item \textbf{Time Availability}: Students have different schedules and availability for participating in code kata battles.
\end{itemize}
\subsection{Assumptions, dependencies and constraints}
\subsubsection{Regulatory policies}
\begin{enumerate}
    \item \textbf{\textit{Data Privacy and Protection:}}
    \begin{itemize}
        \item \textbf{General Data Protection Regulation (GDPR)}: The CKB platform will ask for user personal information like name, surname and email address. Email addresses won’t be used for commercial purposes. Personal information will be processed in compliance with the GDPR.
    \end{itemize}
    \item \textbf{\textit{Intellectual Property Rights:}}
    \begin{itemize}
        \item \textbf{Copyright and Licensing}: The platform must respect copyright and licensing regulations when handling code katas, user submissions, and other content. It should encourage users to respect the intellectual property rights of others and provide mechanisms for reporting copyright violations.
        \item \textbf{Plagiarism Prevention}: The platform should have mechanisms in place to detect and prevent plagiarism in code submissions. This includes educating users on proper citation and attribution practices.
    \end{itemize}
\end{enumerate}

\subsubsection{Domain Assumptions}
The following are the assumptions made for the domain. Such assumptions are properties and/or conditions that the system takes for granted, mostly because they are out of the control of the system itself, and hence need to be verified to assure the correct behavior of CKB.
\\$[D1]$: Users have reliable and consistent access to the internet to participate in code kata battles and interact with the platform.
\\$[D2]$: Users have access to suitable devices, such as computers or mobile devices, with compatible web browsers.
\\$[D3]$: Educators have a basic understanding of the CodeKataBattle platform's features, enabling them to create and manage tournaments and battles effectively. In particular, they have to provide correct tests and rules for projects and badges.
\\$[D4]$: User-provided data, including user profiles, code submissions, and scoring information, is assumed to be accurate and valid. Users are supposed to providing truthful information.
\\$[D5]$: Users will use the platform responsibly, adhering to ethical guidelines and academic integrity. Students are expected to complete code katas independently or within the rules of team collaboration defined by educators.
\\$[D6]$: The platform assumes a high level of system uptime, with minimal downtime, to support user interactions, database storage, and real-time features.
\\$[D7]$: Some platform functionalities may rely on external services, such as email verification, communication tools and Github integrations. These external services are assumed to be available and functional.
\\$[D8]$: Users are responsible for maintaining the security and privacy of their accounts, including choosing strong passwords and keeping their login credentials confidential.
\\$[D9]$: Users are expected to engage in ethical behavior, respecting intellectual property rights, academic integrity, and community guidelines established by the platform.
